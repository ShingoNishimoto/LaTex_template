\documentclass[11pt]{jsarticle}

\usepackage[dvipdfmx]{graphicx}
\usepackage{fancyhdr}
\usepackage{here}
\usepackage{comment}
\usepackage{amsmath,amssymb}
\usepackage{multicol}
\usepackage{mathtools}
\usepackage{multirow}
\usepackage{enumerate}
% 上下に2.5cm、左右に20cmの余白を取る
\usepackage[top=30mm, bottom=30mm, left=20mm, right=20mm]{geometry}

\newif\iffigure
%\figurefaulse
\figuretrue
%select show the figure or not

\makeatletter
\def\@cite#1{\textsuperscript{#1)}}
\def\@biblabel#1{#1)}
\makeatother

\newcommand{\DATE}[3]{#1年#2月#3日} %month, day, year
\newcommand{\TheDay}{\DATE{202X}{XX}{XX}}
\newcommand{\Header}{Header}

\title{Tittle}
\date{\TheDay}
\author{
 山田 太郎
}

%ヘッダの指定:
\pagestyle{fancy}

\begin{document}
\maketitle
\thispagestyle{fancy}
\lhead[\Header]{\Header} % ヘッダ左側
%\chead[偶数ページの引数]{奇数ページの引数} %ヘッダ中央
\rhead[\TheDay]{\TheDay} %ヘッダ右側
%\lfoot[偶数ページの引数]{奇数ページの引数} %フッタ左側
%\cfoot[偶数ページの引数]{奇数ページの引数} %フッタ中央
%\rfoot[偶数ページの引数]{奇数ページの引数} %フッタ右側

\begin{abstract}
  アブスト\\


\end{abstract}

\section{初めに}

\section{背景}
\subsection{先行研究}
\section{手法}
\section{結果}
\section{考察}
\section{結論}

\bibliographystyle{junsrt} % plain, acm, alpha ...
\bibliography{Ref} 
\end{document}